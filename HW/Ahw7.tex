\documentclass[10.5pt]{article}
\usepackage{amsmath,amssymb,amsthm}
\usepackage{amsfonts,bm}
\usepackage{listings}
\usepackage{graphicx}
\usepackage[shortlabels]{enumitem}
\usepackage{tikz}
\usepackage{extramarks}
%\usepackage{enumerate}
\usepackage[margin=1in]{geometry}
\usepackage{fancyhdr}
\usepackage{epsfig}
\usepackage{amsmath}
\usepackage{float}
\usepackage{amssymb}
\usepackage{caption}
\usepackage{subfigure}
\usepackage{graphics}
\usepackage{titlesec}
\usepackage{mathrsfs}
\usepackage{amsfonts}
\usepackage{indentfirst}
\usepackage{fontspec}
\usepackage{dashrule}
\usepackage{ctex}
\usepackage{algpseudocode}
%\usepackage{algorithm}

\renewcommand{\baselinestretch}{1.2}

\usepackage{tikz-qtree}
\usetikzlibrary{graphs}
\tikzset{every tree node/.style={minimum width=2em,draw,circle},
	blank/.style={draw=none},
	edge from parent/.style=
	{draw,edge from parent path={(\tikzparentnode) -- (\tikzchildnode)}},
	level distance=1.2cm} 
\setlength{\parindent}{0pt}
\setlength{\headheight}{13.6pt}
\newcommand\question[1]{\vspace{.2in}\hrule\vspace{0.04in}\textbf{Problem\ #1}\vspace{.4em}\hrule\vspace{.10in}}
\newcommand\Solution{\vspace{.3in}\textbf{Solution:}\vspace{.5em}\hrule\vspace{.08in}\par}
\newcommand\Answer{\vspace{.2in}\textbf{Answer:}\vspace{.5em}\hrule\vspace{.08in}\par}
\newcommand\Proof{\vspace{.3in}\textbf{Proof:}\vspace{.5em}\hrule\vspace{.08in}\par}
\newcommand\minsolution{\vspace{.3in}\textbf{Solution:}\vspace{.4em}\par}
\newcommand\minanswer{\vspace{.2in}\textbf{Answer:}\vspace{.4em}\par}
\newcommand\minproof{\vspace{.3in}\textbf{Proof:}\vspace{.4em}\par}
\renewcommand\part[1]{\vspace{.10in}\textbf{(#1)}}
\newcommand\algorithm{\vspace{.10in}\textbf{Algorithm: }}
\newcommand\correctness{\vspace{.10in}\textbf{Correctness: }}
\newcommand\runtime{\vspace{.10in}\textbf{Running time: }}
\pagestyle{fancyplain}

\setCJKfamilyfont{Song}[AutoFakeBold]{SimSun}
\newcommand*{\Song}{\CJKfamily{Song}}




\newcommand{\horrule}[1]{\rule{\linewidth}{#1}}

\title{
	\normalfont \normalsize
	\begin{figure}[!h]
	\centering
	\includegraphics[width=4.8in, keepaspectratio]{logo_red.pdf}\\[1cm]
		%\caption{}
	\end{figure}
	%\huge{\textsc{ShanghaiTech University}} \\ [8pt]
	\horrule{0.5pt} \\[0.4cm]
	\Huge SI140 Probability \& Mathematical Statistics\\[0.4cm]
	\LARGE Homework 7\\
	\horrule{2pt} \\[1.5cm]
}

\author{\Song{\huge\textbf{陈昱聪}}\\[0.2cm]Chen Yucong\ ><E<>N\\[4.5cm]\textbf{Student ID: 2019533079}\\[0.2cm] 
\textbf{Email:}\ {\ttfamily chenyc@shanghaitech.edu.cn}\\[0.8cm] \LARGE\textsc{School of Information Science and Technology}\\[0.63cm]
\texttt{$\circledcirc$ Group\#2\ (TA:曾理)}}
\date{}


\pagestyle{fancy}
\lhead{SI140 Probability \& Mathematical Statistics}
\chead{\textbf{Homework 7\ }}
\rhead{陈昱聪\,2019533079\ \,Due:\,11:59\,am, $09^{\text{th}}$ Nov.}
\cfoot{\thepage}
\renewcommand{\headrulewidth}{0.4pt}


\fancypagestyle{firstpage}
{
	\renewcommand{\headrulewidth}{0pt}
	\fancyhf{}
	\fancyfoot[C]{\thepage}
}


\newcounter{ProblemCounter}
\newcounter{oldvalue}
\newcommand{\problem}[2][-1]{
	\setcounter{oldvalue}{\value{secnumdepth}}
	\setcounter{secnumdepth}{0}
	\ifnum#1>-1
	\setcounter{ProblemCounter}{0}
	\else
	\stepcounter{ProblemCounter}
	\fi
	\section{Problem \arabic{ProblemCounter}: #2}
	\setcounter{secnumdepth}{\value{oldvalue}}
}
\newcommand{\subproblem}[1]{
	\setcounter{oldvalue}{\value{section}}
	\setcounter{section}{\value{ProblemCounter}}
	\subsection{#1}
	\setcounter{section}{\value{oldvalue}}
}

\setmonofont{Consolas}
\definecolor{blve}{rgb}{0.3372549 , 0.61176471, 0.83921569}
\definecolor{gr33n}{rgb}{0.29019608, 0.7372549 , 0.64705882}
\makeatletter
\lst@InstallKeywords k{class}{classstyle}\slshape{classstyle}{}ld
\makeatother
\lstset{language=C++,
	basicstyle=\ttfamily,
	keywordstyle=\color{blve}\ttfamily,
	stringstyle=\color{red}\ttfamily,
	commentstyle=\color{green}\ttfamily,
	morecomment=[l][\color{magenta}]{\#},
	classstyle = \bfseries\color{gr33n}, 
	tabsize=4
}


\begin{document}
	
\maketitle
\thispagestyle{firstpage}
\thispagestyle{empty}
\setcounter{page}{0}

\pagebreak

\question{5.8}
	\Solution{}
\begin{enumerate}[(a)]
	\item \begin{align*}
		F(x) = \int_{0}^x\,f(t)\,\mathrm{d}t=\int_{0}^x\,12t^2(1-t)\,\mathrm{d}t=4x^3-3x^4,\quad \text{for}\ 0<x<1
	\end{align*}
	\vspace{1cm}\item 
	$$P(0<X<1/2) = F(1/2) = \frac{5}{16}$$\vspace{1cm}\item 
	\begin{align*}
		E(X) = \int_{0}^1\,xf(x)\,\mathrm{d}x = \int_{0}^1\,12x^3(1-x)\,\mathrm{d}x=\frac{3}{5}
	\end{align*}
	To get variance, we have: $$E(X^2) = E(X) = \int_{0}^1\,x^2f(x)\,\mathrm{d}x = \frac{2}{5}$$
	So that $$V(X) = E(X^2) - E^2(X) = \frac{1}{25}$$
\end{enumerate}
   
\vspace{1cm}


\question{5.14}
	\Solution{}
	\begin{align*}
		F(x) = P(X\leqslant x) = P(U_1\leqslant x, U_2\leqslant x,\dots, U_n\leqslant x) = P(U_1\leqslant x)P(U_2\leqslant x)\dots P(U_n\leqslant x) = x^n
	\end{align*}
	So the PDF of $X$ is:
	$$f(x) = \frac{\mathrm{d} F(x)}{\mathrm{d} x} = nx^{n - 1}$$
	And we get:
	$$EX = \int_{0}^1\,xf(x)\,\mathrm{d}x = \frac{n}{n+1}$$


\pagebreak

\question{5.31}
	\Solution{}
\begin{enumerate}[(a)]
	\item For $y<0$, since $Y\geqslant0$ so $F(y) = 0$ in this case.
	For $y\geqslant0$, $P(Y\leqslant y) = P(-y\leqslant x \leqslant y) = \Phi(\frac{y - \mu}{\sigma}) - \Phi(\frac{-y - \mu}{\sigma})$, so that:
	\begin{equation*}
		F(y) = 
		\begin{cases}
			\Phi(\frac{y - \mu}{\sigma}) - \Phi(\frac{-y - \mu}{\sigma})&\quad\ y\geqslant0\\[6pt]
			0&\quad\ y<0
		\end{cases}
	\end{equation*}\vspace{1cm}
	\item For $y<0, f(y)=0$; 
	for $y>0, f(y) = \frac{\mathrm{d} F(y)}{\mathrm{d} y}=\varphi(\frac{y - \mu}{\sigma})\frac{1}{\sigma}+\varphi(\frac{-y - \mu}{\sigma})\frac{1}{\sigma}$; 
	But since $f(y)$ has no limit at 0, we can't define the value here, let's just say that $f(0) = c$ where $c$ can be any real number. So that:
	\begin{equation*}
		f(y) = 
		\begin{cases}
			\frac{1}{\sigma\sqrt{2\pi}}\cdot(e^{-\frac{(y-\mu)^2}{2\sigma^2}}+e^{-\frac{(y+\mu)^2}{2\sigma^2}})&\quad\ y\geqslant0\\[6pt]
			c&\quad\ y=0\\[6pt]
			0&\quad\ y<0
		\end{cases}
	\end{equation*}\vspace{1cm}
	\item Not continuous, since the left limit does not equal to the right limit. There is no problem to using it to find probabilities because the values of PDF at finite number of $y$s won't affect area under the graph of PDF.
\end{enumerate}
	
\vspace{0.2cm}


\question{5.45}
\Solution{}
\begin{enumerate}[(a)]
	\item It is easy to notice that if the number of failures is $1$, then $G = 1$ and $T = \Delta t$, if $2$ then $G = 2$ and $T = 2\Delta t\dots$, we found that $\Delta t\cdot G = T$.\vspace{0.2cm}
	\item $$F(t) = P(T \leqslant t) = 1-P(T>t) = 1 - P(G > \frac{t}{\Delta t}) = 1 - (1-\lambda\Delta t)^{\frac{t}{\Delta t}}$$\vspace{0.2cm}
	\item Using the compound interest limit, we have $$\lim_{\Delta t \to 0}\,\left[1 - (1-\lambda\Delta t)^{\frac{t}{\Delta t}}\right] = 1 - e^{-\lambda t},\quad \text{for all fixed}\ t>0$$
	That is right as the Expo$(\lambda)$ CDF.
\end{enumerate}




\pagebreak

\question{5.55}
\Solution{}
\begin{align*}
	\text{MSE}(T) - \text{Var}(T) &= E(T^2+\theta^2-2T\theta) - (E(T^2) - E^2(T))\\[8pt]
	&=E(T^2)+E(\theta^2)-E(2T\theta) - E(T^2)+E^2(T)\\[8pt]
	&=E(T^2)+\theta^2-2\theta E(T)-E(T^2)+E^2(T)\\[8pt]
	&=E^2(T)-2\theta E(T)+\theta^2\\[8pt]
	&=(b(T))^2
\end{align*}
That is MSE$(T) = $Var$(T)+(b(T))^2$.




\question{6.25}
\Solution{}
\begin{enumerate}[(a)]
	\item \begin{align*}
		P(Y>s+t|Y>s) &= \frac{P(Y>s+t,\, Y>s)}{P(Y>s)} = \frac{P(Y>s+t)}{P(Y>s)}\\[8pt]
		&= \frac{P(X>\sqrt[3]{s+t})}{P(X>\sqrt[3]{s})}=\frac{1 - P(X\leqslant\sqrt[3]{s+t})}{1 - P(X\leqslant\sqrt[3]{s})}= \frac{e^{-\sqrt[3]{s+t}}}{e^{-\sqrt[3]{s}}}\neq e^{-\sqrt[3]{t}}
\end{align*}
So $Y$ does \textbf{not} have the memoryless property.\vspace{0.2cm}
	\item Since $F(y) = P(Y\leqslant y) = P(X \leqslant \sqrt[3]{y})$, we have:\begin{align*}
		f(y) &= \frac{\mathrm{d} F(y)}{\mathrm{d} y} = \frac{\mathrm{d} (1 - e^{-y^{1/3}})}{\mathrm{d} y} = \frac{y^{-2/3}\cdot e^{-y^{1/3}}}{3}\\[10pt]
		E(Y^n) 
		&= \int_{0}^{+\infty} (y^n\cdot\frac{y^{-2/3}\cdot e^{-y^{1/3}}}{3})\, \mathrm{d}y = \frac{1}{3}\int_{0}^{+\infty} (y^{n - 2/3}\cdot e^{-y^{1/3}})\, \mathrm{d}y\\[8pt]
		&= \frac{1}{3}\int_{0}^{+\infty} (u^{3n - 2}\cdot e^{-u})\, \mathrm{d}u^3 = \int_{0}^{+\infty} (u^{3n}\cdot e^{-u})\, \mathrm{d}u = \Gamma(3n+1) = (3n)!
	\end{align*}
	So we have $E(Y) = 3! = 6,\quad E(Y^2) = (3\cdot2)! = 720,\quad V(Y) = E(Y^2) - E^2(Y) = 684$\vspace{0.2cm}
	\item $$M(t) = \sum_{n = 0}^\infty M^{(n)}(0)\frac{t^n}{n!} = \sum_{n = 0}^\infty E(Y^n)\frac{t^n}{n!} = \sum_{n = 0}^\infty \frac{(3n)!\cdot t^n}{n!}$$
	It does not converge at any non-zero point since $t^n = \omega(\frac{n!}{(3n)!})$ when $t>0$ using Stirling's formula.
	So the MGF does \textbf{not} exist.
\end{enumerate}

\end{document}