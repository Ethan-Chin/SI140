\documentclass[10.5pt]{article}
\usepackage{amsmath,amssymb,amsthm}
\usepackage{amsfonts,bm}
\usepackage{listings}
\usepackage{graphicx}
\usepackage[shortlabels]{enumitem}
\usepackage{tikz}
\usepackage{extramarks}
%\usepackage{enumerate}
\usepackage[margin=1in]{geometry}
\usepackage{fancyhdr}
\usepackage{epsfig}
\usepackage{amsmath}
\usepackage{float}
\usepackage{amssymb}
\usepackage{caption}
\usepackage{subfigure}
\usepackage{graphics}
\usepackage{titlesec}
\usepackage{mathrsfs}
\usepackage{amsfonts}
\usepackage{indentfirst}
\usepackage{fontspec}
\usepackage{dashrule}
\usepackage{ctex}
\usepackage{algpseudocode}
%\usepackage{algorithm}

\renewcommand{\baselinestretch}{1.2}

\usepackage{tikz-qtree}
\usetikzlibrary{graphs}
\tikzset{every tree node/.style={minimum width=2em,draw,circle},
	blank/.style={draw=none},
	edge from parent/.style=
	{draw,edge from parent path={(\tikzparentnode) -- (\tikzchildnode)}},
	level distance=1.2cm} 
\setlength{\parindent}{0pt}
\setlength{\headheight}{13.6pt}
\newcommand\question[1]{\vspace{.2in}\hrule\vspace{0.04in}\textbf{Problem\ #1}\vspace{.4em}\hrule\vspace{.10in}}
\newcommand\Solution{\vspace{.3in}\textbf{Solution:}\vspace{.5em}\hrule\vspace{.08in}\par}
\newcommand\Answer{\vspace{.2in}\textbf{Answer:}\vspace{.5em}\hrule\vspace{.08in}\par}
\newcommand\Proof{\vspace{.3in}\textbf{Proof:}\vspace{.5em}\hrule\vspace{.08in}\par}
\newcommand\minsolution{\vspace{.3in}\textbf{Solution:}\vspace{.4em}\par}
\newcommand\minanswer{\vspace{.2in}\textbf{Answer:}\vspace{.4em}\par}
\newcommand\minproof{\vspace{.3in}\textbf{Proof:}\vspace{.4em}\par}
\renewcommand\part[1]{\vspace{.10in}\textbf{(#1)}}
\newcommand\algorithm{\vspace{.10in}\textbf{Algorithm: }}
\newcommand\correctness{\vspace{.10in}\textbf{Correctness: }}
\newcommand\runtime{\vspace{.10in}\textbf{Running time: }}
\pagestyle{fancyplain}

\setCJKfamilyfont{Song}[AutoFakeBold]{SimSun}
\newcommand*{\Song}{\CJKfamily{Song}}




\newcommand{\horrule}[1]{\rule{\linewidth}{#1}}

\title{
	\normalfont \normalsize
	\begin{figure}[!h]
	\centering
	\includegraphics[width=4.8in, keepaspectratio]{logo_red.pdf}\\[1cm]
		%\caption{}
	\end{figure}
	%\huge{\textsc{ShanghaiTech University}} \\ [8pt]
	\horrule{0.5pt} \\[0.4cm]
	\Huge SI140 Probability \& Mathematical Statistics\\[0.4cm]
	\LARGE Homework 5\\
	\horrule{2pt} \\[1.5cm]
}

\author{\Song{\huge\textbf{陈昱聪}}\\[0.2cm]Chen Yucong\ ><E<>N\\[4.5cm]\textbf{Student ID: 2019533079}\\[0.2cm] 
\textbf{Email:}\ {\ttfamily chenyc@shanghaitech.edu.cn}\\[0.8cm] \LARGE\textsc{School of Information Science and Technology}\\[0.63cm]
\texttt{$\circledcirc$ Group\#2\ (TA:曾理)}}
\date{}


\pagestyle{fancy}
\lhead{SI140 Probability \& Mathematical Statistics}
\chead{\textbf{Homework 5}}
\rhead{陈昱聪\ 2019533079\quad Due: 11:59\,am $19^{\text{th}}$ Oct.}
\cfoot{\thepage}
\renewcommand{\headrulewidth}{0.4pt}


\fancypagestyle{firstpage}
{
	\renewcommand{\headrulewidth}{0pt}
	\fancyhf{}
	\fancyfoot[C]{\thepage}
}


\newcounter{ProblemCounter}
\newcounter{oldvalue}
\newcommand{\problem}[2][-1]{
	\setcounter{oldvalue}{\value{secnumdepth}}
	\setcounter{secnumdepth}{0}
	\ifnum#1>-1
	\setcounter{ProblemCounter}{0}
	\else
	\stepcounter{ProblemCounter}
	\fi
	\section{Problem \arabic{ProblemCounter}: #2}
	\setcounter{secnumdepth}{\value{oldvalue}}
}
\newcommand{\subproblem}[1]{
	\setcounter{oldvalue}{\value{section}}
	\setcounter{section}{\value{ProblemCounter}}
	\subsection{#1}
	\setcounter{section}{\value{oldvalue}}
}

\setmonofont{Consolas}
\definecolor{blve}{rgb}{0.3372549 , 0.61176471, 0.83921569}
\definecolor{gr33n}{rgb}{0.29019608, 0.7372549 , 0.64705882}
\makeatletter
\lst@InstallKeywords k{class}{classstyle}\slshape{classstyle}{}ld
\makeatother
\lstset{language=C++,
	basicstyle=\ttfamily,
	keywordstyle=\color{blve}\ttfamily,
	stringstyle=\color{red}\ttfamily,
	commentstyle=\color{green}\ttfamily,
	morecomment=[l][\color{magenta}]{\#},
	classstyle = \bfseries\color{gr33n}, 
	tabsize=4
}


\begin{document}
	
\maketitle
\thispagestyle{firstpage}
\thispagestyle{empty}
\setcounter{page}{0}

\pagebreak

\question{2.14}
	\Solution{}
	\begin{enumerate}[(a)]
		\item \begin{align*}
			P(X\geqslant1) &= 1 - P(X = 0) = 1 - \frac{e^{-\lambda}\lambda^0}{0!} = 1 - e^{-\lambda}\\[6pt]
			P(X\geqslant2) &= P(X\geqslant1) - P(X = 1)= 1 - (1+\lambda)e^{-\lambda}
		\end{align*}
		\item \begin{align*}
			P(X=k|X\geqslant1) &= \frac{P(X\geqslant1|X=k)P(X=k)}{P(X\geqslant1)}=\frac{P(X=k)}{P(X\geqslant1)} = \frac{e^{-\lambda}\lambda^k}{(1-e^{-\lambda})k!},\quad k = 1, 2, \dots\\[6pt]
		\end{align*}
	\end{enumerate}
   

\vspace{3cm}

\question{3.31}
	\Solution{}
	\begin{enumerate}[(a)]
		\item The number of correct guesses is $X$. We know that $X\sim \text{HGeom}(3, 3, 3)$.
		We have:
			$$P(X \geqslant 2) = \frac{\binom{3}{2}\binom{3}{1}}{\binom{6}{3}}+\frac{\binom{3}{3}\binom{3}{0}}{\binom{6}{3}} = \frac{1}{2}$$
		\item Let $A = \{\text{She claims that the cup was milk-first.}\},\quad B = \{\text{The cup is milk-first.}\}$
		\begin{align*}
			posterior\ odds 
			= \frac{P(B|A)}{P(B^c|A)}
			= \frac{P(A|B)P(B)}{P(A|B^c)P(B^c)}
			= \frac{p_1}{1-p_2}
		\end{align*}
	\end{enumerate}

\pagebreak

\question{3.37}
	\Solution{}
	\begin{enumerate}[(a)]
		\item When even errors occured, that is 2 and 4, it will be undetected.
		$$P(\text{Errors going undetected}) = \binom{5}{2}(0.1)^2(0.9)^3+\binom{5}{4}(0.1)^4(0.9)^1 = 0.7335$$
		\vspace{0.5cm}
		\item When even errors occured, it will be undetected.
		$$P(\text{Errors going undetected}) = \sum_{k\ \text{is even and}\ 2\leqslant k\leqslant n} \binom{n}{k}p^k(1-p)^{n-k}$$
		\vspace{0.5cm}
		\item \begin{gather*}
			a+b = 1;\qquad a-b = \sum_{k = 0}^n \binom{n}{k}(-p)^k(1-p)^{n-k} = (1-2p)^n\\[6pt]
			\Rightarrow a = \frac{1+(1-2p)^n}{2};\qquad b = \frac{1 - (1-2p)^n}{2}\\[6pt]
			\Rightarrow P(\text{Errors going undetected}) = a - \binom{n}{0}p^0(1-p)^n = \frac{1+(1-2p)^n}{2}-(1-p)^n
		\end{gather*}
	\end{enumerate}

\vspace{1cm}

\question{3.44}
\Solution{}
\begin{enumerate}[(a)]
	\item \vspace{1cm}\begin{align*}
		P(X\oplus Y = 1) = P(X = 1|Y = 1)P(Y = 1)+P(X = 0|Y = 0)P(Y = 0) = \frac{1}{2}\\[6pt]
	\end{align*}
		So that $$X\oplus Y\sim\text{Bern}(\frac{1}{2})$$

	\item Let $\qquad A = \{X\oplus Y = 1\};\qquad B = \{X = 1\};\qquad C = \{Y = 1\}$
	\begin{align*}
		P(A\cap B) = P(B\cap C^c) = \frac{p}{2} = P(A)P(B)
	\end{align*}
	So that $A$ and $B$ are independent. That is $A$ and $B^c$, $A^c$ and $B$, $A^c$ and $B^c$ are independent. That is $X\oplus Y$ and $X$ are independent whatever $p$ is.

	\begin{align*}
		P(A\cap C) = P(B^c\cap C) = \frac{1-p}{2}
	\end{align*}
	When $p = \frac{1}{2}$, $P(A\cap C) = P(A)P(C) = \frac{1}{4}$, then like the reason above, $X\oplus Y$ and $Y$ are independent. If $p\neq \frac{1}{2}$, then $P(A\cap C) \neq P(A)P(C)$, so they are not independent.
	\vspace{1cm}
	\item When $J$ is of size $1$, it is obvious that $Y_{J_1}\sim \text{Bern}(\frac{1}{2})$. When $J$ is of size $2$, we know that $Y_{J_2} = Y_{J_1}\oplus X_i$ for any other $i$. Since $X_i\sim \text{Bern}(\frac{1}{2})$, using the conclusion in (a), $Y_{J_2}\sim \text{Bern}(\frac{1}{2})$.
	By using induction, $Y_J\sim \text{Bern}(\frac{1}{2})$ for each nonempty subset $J$.

	Using the explanation in Hint, we know that for $Y_J$ and $Y_K$, $A$, $B$ and $C$ are independent. If $J\cap K = \emptyset$,
	\begin{align*}
		p(Y_J = 0\cap Y_K = 0) = P(Y_J = 0)P(Y_K = 0)
	\end{align*}
	Because they don't relate to any ame $X_i$, so that they are independent.

	If $A \neq \emptyset$,
	\begin{align*}
		p(Y_J = 0\cap Y_K = 0) 
		&= P(Y_J = 0\cap Y_K = 0|A = 1)P(A=1)+(Y_J = 0\cap Y_K = 0|A=0)P(A=0)\\[6pt]
		&= \frac{1}{2}\cdot\frac{1}{2}\cdot\frac{1}{2}+\frac{1}{2}\cdot\frac{1}{2}\cdot\frac{1}{2} = \frac{1}{4} = P(Y_J = 0)P(Y_K = 0)\\[6pt]
	\end{align*}So that they are independent. That is, $Y_J$ are pairwise independent.
	
	But since $P(Y_{1} = 0\cap Y_{2} = 0\cap Y_{\{1, 2\}} = 0) = \frac{1}{4}\neq P(Y_{1} = 0)\cap P(Y_{2} = 0)P\cap(Y_{\{1, 2\}} = 0) = \frac{1}{8}$, we know that $Y_J$ are not independent.
\end{enumerate}

\pagebreak

\question{3.46}
\Solution{}
\begin{enumerate}[(a)] 
	\item If $A$ starting with $1$ dollars, and finally reaches to 0 dollars, that means $A$ must wins twice more than failures (\# failures is exactly 1 greater than \# wins). So that's typically the original problem wich $p_1$ is the probability of the original problem.
	\vspace{1cm}
	\item From the constrains, $p_k = \frac{1}{2}p_{k-1}+\frac{1}{2}p_{k+2}$ (It makes sense that the notation could be negative). The Characteristic equation is $\lambda^3-2\lambda+1 = 0$. Find that:
	$$\lambda_1 = 1;\quad\lambda_2 = \frac{-1-\sqrt{5}}{2};\quad\lambda_3 = \frac{-1+\sqrt{5}}{2}$$
	So $$p_k = a+b\left(\frac{-1-\sqrt{5}}{2}\right)^k+c\left(\frac{-1+\sqrt{5}}{2}\right)^k$$
	We know that $$p_0 = 1;\quad \lim_{k \to \infty} p_k = 0$$
	So that $a = b = 0,\quad c = 1$
	That is $$p_k = \left(\frac{-1+\sqrt{5}}{2}\right)^k$$
	\vspace{1cm}
	\item The original problem is to consider the $p_1$.
	So $$P_{\text{original}} = p_1 = \frac{-1+\sqrt{5}}{2}$$
\end{enumerate}

\pagebreak

\question{3.47}
\Solution{}
\begin{enumerate}[(a)]
	\item If $n\leqslant m$, then $P = 1$; If $n > 2m$, then $P = 0$. Now consider the case that $m < n \leqslant 2m$.
	\begin{align*}
		P 
		&= \sum_{n - m}^m \binom{n}{k}\left(\frac{1}{2}\right)^k\left(\frac{1}{2}\right)^{n - k}\\[6pt]
		&=\sum_{k = 0}^{m} \binom{n}{k}\left(\frac{1}{2}\right)^k\left(\frac{1}{2}\right)^{n - k} - \sum_{k = 0}^{n - m - 1} \binom{n}{k}\left(\frac{1}{2}\right)^k\left(\frac{1}{2}\right)^{n - k}\\[6pt]
		&=\text{pbinom}(m, n, \frac{1}{2}) - \text{pbinom}(n - m - 1, n, \frac{1}{2})
	\end{align*}
	\item Using python to sole this problem. The result is $m = 8$ for $n = 10$, $m = 60$ for $n = 100$, $m = 531$ for $n = 1000$, $m = 5098$ for $n = 10000$.
	The code is as below:
	\begin{lstlisting}[language = python]
		import random

		def copyPaper(n:int):
			for m in range(int(n/2), n):
				suc = 0
				
				for i in range(1000):
					tray0 = tray1 = m
		
					for j in range(n):
						if random.randint(0, 1):
							tray1 -= 1
						else:
							tray0 -= 1
		
						if tray0 < 0 or tray1 < 0:
							break
					
					else:
						suc += 1
		
				if (suc / 1000) >= 0.95:
					return m
			
		n = int(input("Please enter the n: "))
		
		print("The smallest m is: ", copyPaper(n))
	\end{lstlisting}
\end{enumerate}


\end{document}