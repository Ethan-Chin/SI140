\documentclass[10.5pt]{article}
\usepackage{amsmath,amssymb,amsthm}
\usepackage{amsfonts,bm}
\usepackage{listings}
\usepackage{graphicx}
\usepackage[shortlabels]{enumitem}
\usepackage{tikz}
\usepackage{extramarks}
%\usepackage{enumerate}
\usepackage[margin=1in]{geometry}
\usepackage{fancyhdr}
\usepackage{epsfig}
\usepackage{amsmath}
\usepackage{float}
\usepackage{amssymb}
\usepackage{caption}
\usepackage{subfigure}
\usepackage{graphics}
\usepackage{titlesec}
\usepackage{mathrsfs}
\usepackage{amsfonts}
\usepackage{indentfirst}
\usepackage{fontspec}
\usepackage{dashrule}
\usepackage{ctex}
\usepackage{algpseudocode}
%\usepackage{algorithm}

\renewcommand{\baselinestretch}{1.2}

\usepackage{tikz-qtree}
\usetikzlibrary{graphs}
\tikzset{every tree node/.style={minimum width=2em,draw,circle},
	blank/.style={draw=none},
	edge from parent/.style=
	{draw,edge from parent path={(\tikzparentnode) -- (\tikzchildnode)}},
	level distance=1.2cm} 
\setlength{\parindent}{0pt}
\setlength{\headheight}{13.6pt}
\newcommand\question[1]{\vspace{.2in}\hrule\vspace{0.04in}\textbf{Problem\ #1}\vspace{.4em}\hrule\vspace{.10in}}
\newcommand\Solution{\vspace{.3in}\textbf{Solution:}\vspace{.5em}\hrule\vspace{.08in}\par}
\newcommand\Answer{\vspace{.2in}\textbf{Answer:}\vspace{.5em}\hrule\vspace{.08in}\par}
\newcommand\Proof{\vspace{.3in}\textbf{Proof:}\vspace{.5em}\hrule\vspace{.08in}\par}
\newcommand\minsolution{\vspace{.3in}\textbf{Solution:}\vspace{.4em}\par}
\newcommand\minanswer{\vspace{.2in}\textbf{Answer:}\vspace{.4em}\par}
\newcommand\minproof{\vspace{.3in}\textbf{Proof:}\vspace{.4em}\par}
\renewcommand\part[1]{\vspace{.10in}\textbf{(#1)}}
\newcommand\algorithm{\vspace{.10in}\textbf{Algorithm: }}
\newcommand\correctness{\vspace{.10in}\textbf{Correctness: }}
\newcommand\runtime{\vspace{.10in}\textbf{Running time: }}
\pagestyle{fancyplain}

\setCJKfamilyfont{Song}[AutoFakeBold]{宋体-简 细体}
\newcommand*{\Song}{\CJKfamily{Song}}




\newcommand{\horrule}[1]{\rule{\linewidth}{#1}}

\title{
	\normalfont \normalsize
	\begin{figure}[!h]
	\centering
	\includegraphics[width=4.8in, keepaspectratio]{logo_red.pdf}\\[1cm]
		%\caption{}
	\end{figure}
	%\huge{\textsc{ShanghaiTech University}} \\ [8pt]
	\horrule{0.5pt} \\[0.4cm]
	\Huge SI140 Probability \& Mathematical Statistics\\[0.4cm]
	\LARGE Homework 10\\
	\horrule{2pt} \\[1.5cm]
}
 
\author{\Song{\huge\textbf{陈昱聪}}\\[0.2cm]Chen Yucong\ ><E<>N\\[4.5cm]\textbf{Student ID: 2019533079}\\[0.2cm] 
\textbf{Email:}\ {\ttfamily chenyc@shanghaitech.edu.cn}\\[0.8cm] \LARGE\textsc{School of Information Science and Technology}\\[0.63cm]
\texttt{$\circledcirc$ Group\#2\ (TA:曾理)}}
\date{}


\pagestyle{fancy}
\lhead{SI140 Probability \& Mathematical Statistics}
\chead{\textbf{Homework 10\ }}
\rhead{陈昱聪\,2019533079\ \,Due:\,11:59\,am, $7^{\text{th}}$ Dec.}
\cfoot{\thepage}
\renewcommand{\headrulewidth}{0.4pt}


\fancypagestyle{firstpage}
{
	\renewcommand{\headrulewidth}{0pt}
	\fancyhf{}
	\fancyfoot[C]{\thepage}
}


\newcounter{ProblemCounter}
\newcounter{oldvalue}
\newcommand{\problem}[2][-1]{
	\setcounter{oldvalue}{\value{secnumdepth}}
	\setcounter{secnumdepth}{0}
	\ifnum#1>-1
	\setcounter{ProblemCounter}{0}
	\else
	\stepcounter{ProblemCounter}
	\fi
	\section{Problem \arabic{ProblemCounter}: #2}
	\setcounter{secnumdepth}{\value{oldvalue}}
}
\newcommand{\subproblem}[1]{
	\setcounter{oldvalue}{\value{section}}
	\setcounter{section}{\value{ProblemCounter}}
	\subsection{#1}
	\setcounter{section}{\value{oldvalue}}
}

\setmonofont{Helvetica}
\definecolor{blve}{rgb}{0.3372549 , 0.61176471, 0.83921569}
\definecolor{gr33n}{rgb}{0.29019608, 0.7372549 , 0.64705882}
\makeatletter
\lst@InstallKeywords k{class}{classstyle}\slshape{classstyle}{}ld
\makeatother
\lstset{language=C++,
	basicstyle=\ttfamily,
	keywordstyle=\color{blve}\ttfamily,
	stringstyle=\color{red}\ttfamily,
	commentstyle=\color{green}\ttfamily,
	morecomment=[l][\color{magenta}]{\#},
	classstyle = \bfseries\color{gr33n}, 
	tabsize=4
}


\begin{document}
	
\maketitle
\thispagestyle{firstpage}
\thispagestyle{empty}
\setcounter{page}{0}



\question{7.66}
\Solution{}
\begin{enumerate}[(a)]
	\item $$p_1=\frac{a+b}{a+b+c+d}\qquad p_2 = \frac{c}{a+b+c+d}\qquad p_3 = \frac{d}{a+b+c+d}$$
	
	\begin{align*}
		&P(X=n_1,Y=n_2,Z=n_3)\\[8pt]
		&= \frac{n!}{n_1!n_2!n_3!}p_1^{n_1}p_2^{n_2}p_3^{n_3}\\[8pt]
		&=\frac{n!}{n_1!n_2!n_3!}\left(\frac{a+b}{a+b+c+d}\right)^{n_1}\left(\frac{c}{a+b+c+d}\right)^{n_2}\left(\frac{d}{a+b+c+d}\right)^{n_3}
	\end{align*}
For all non-negative integers such that $n_1+n_2+n_3 = n\\$ $P(X=n_1,Y=n_2,Z=n_3)=0$ otherwise.\vspace{0.5cm}
	\item This could be seen as a ``HGeom" distribution with $3$ categories, thus
	\begin{align*}
		P(X=n_1,Y=n_2,Z=n_3) = \frac{\binom{a+b}{n_1}\binom{c}{n_2}\binom{d}{n_3}}{\binom{a+b+c+d}{n}}
	\end{align*}
	For all non-negative integers such that $n_1+n_2+n_3 = n\\$ $P(X=n_1,Y=n_2,Z=n_3)=0$ otherwise.
\end{enumerate}
\pagebreak
\question{7.70}
\Solution{}
\begin{enumerate}[(a)]
	\item \begin{align*}
		P(X_1=n_1, X_2=n_2, X_3=n_3)
		&=\frac{n!}{n_1!n_2!n_3!}(p^2)^{n_1}\left[2p(1-p)\right]^{n_2}\left[(1-p)^2\right]^{n_3}\\[8pt]
		&=\frac{n!}{n_1!n_2!n_3!}2^{n_2}p^{2n_1+n_2}(1-p)^{n_2+2n_3}
	\end{align*}
	For all non-negative integers such that $n_1+n_2+n_3 = n\\$ $P(X_1=n_1, X_2=n_2, X_3=n_3)=0$ otherwise.\vspace{0.5cm}
	\item Let $X$ be \#have an A and with $p(2-p)$ in success and $q_{B}=(1-p)^2$ in failure.
	Since $p_c+q_c=1$, this is binomial given $X\sim Bin(n, p(2-p))$.\vspace{0.5cm}
	\item Let Y be \#A in the $2n$ genes. Since the frequency of A in population is $p$, we get $Y$ is binomial given $Y\sim Bin(2n, p)$\vspace{0.5cm}
	\item Find the estimator of $p$ by caculating the sample proportion. We know that \#A in the sample of size $n$ is $2X_1+X_2$, and the total number of genes is $2(X_1+X_2+X_3)=2n$.
	We get \begin{align*}
		\hat{p} =\frac{2X_1+X_2}{2(X_1+X_2+X_3)} =\frac{2X_1+X_2}{2n}
	\end{align*}\vspace{0.5cm}
	\item Since we can't get any information within $X_1$ and $X_2$, now we consider $X_3$. Find the estimator of $p$ by caculating the sample proportion.
	We get \begin{align*}
	(1-\hat{p})^2 &=\frac{X_3}{(X_1+X_2+X_3)} =\frac{X_3}{n}\\[8pt]
	\Rightarrow\hat{p} &= 1-\sqrt{\frac{X_3}{n}}
	\end{align*}
\end{enumerate}
\pagebreak

\question{7.71}
\Solution{}
\begin{enumerate}[(a)]
	\item For any arbitrary linear combination of $X+Y$ and $X-Y$:$$t(X+Y)+s(X-Y)$$
	can also be written as a linear combination of $X$ and $Y$:$$(t+s)X+(t-s)Y$$
	which is Normal since $(X, Y)$ is Bivariate Normal. So $(X+Y, X-Y)$ is also Bivariate Normal.\vspace{1cm}
	\item Cov$(X+Y, X-Y)=$Var$(X)-$Var$(Y) = 0$, so $X+Y$ and $X-Y$ are independent.
	\begin{align*}
		\text{Var}(X+Y) = \text{Var}(X) + \text{Var}(Y) + 2\rho = 2+2\rho\\[6pt]
		\text{Var}(X-Y) = \text{Var}(X) + \text{Var}(Y) - 2\rho = 2-2\rho\\[6pt]
		\Rightarrow (X+Y)\sim\mathcal{N}(0, 2+2\rho)\qquad(X-Y)\sim\mathcal{N}(0, 2-2\rho)\\[6pt]
		\text{So the joint PDF is: } f(a, b)=\frac{1}{4\pi\sqrt{1-\rho^2}}e^{-\frac{1}{4}\left(a^2/(1+\rho)+b^2/(1-\rho)\right)}
	\end{align*}
\end{enumerate}

\pagebreak

\question{7.75}
\Solution{}
\begin{enumerate}[(a)]
	\item For any arbitrary linear combination of $X$, $Y$ and $X+Y$:$$t_1X+t_2Y+t_3(X+Y)$$
	can also be written as a linear combination of $X$ and $Y$:$$(t_1+t_3)X+(t_2+t_3)Y$$
	which is Normal since $(t_1+t_3)X+(t_2+t_3)Y$ is a linear combination of two i.i.d. r.v.s. of Normal. So $(X, Y, X+Y)$ is Multivariate Normal.\vspace{1cm}
	\item Let $t_1 = t_2 = -1, t_3 = 1$, since $P(-X-Y+SX+SY = 0) = P(S = 1) = \frac{1}{2}$, this combination is not a continuous r.v. so that $(X, Y, SX+SY)$ is \textbf{not} Multivariate Normal.\vspace{1cm}
	\item Find the PDF of $Z = t_1SX+t_2SY$,$\ W=t_1X+t_2Y$\begin{align*}
		f_Z(a)=& f_Z(a|S = 1)P(S = 1)+f_Z(a|S = -1)P(S = -1)\\[8pt]
		=& \frac{1}{2}\, f_Z(a|S = 1)+\frac{1}{2}\, f_Z(-a|S = -1)\\[8pt]
		=& \frac{1}{2}\, f_W(a)+\frac{1}{2}\, f_W(-a)\\[8pt]
		=& f_W(a)\qquad(\text{By considering $W\sim\mathcal{N}(0, t_1^2\, t_2^2)$, it is symmetric about 0})
	\end{align*}
	So for any $t_1$ and $t_2$, $t_1SX+t_2SY$ is just the same as $t_1X+t_2Y$, since the latter one is Normal, we get $t_1SX+t_2SY$ is Normal, so $(SX, SY)$ is Multivariate Normal.
\end{enumerate}
\pagebreak
\question{7.76}
\Solution{}
First to find $c$ such that Cov$(Y-cX, X)=0$:
\begin{align*}
	\text{Cov}(Y-cX, X) &= \text{Cov}(Y,X)-c\text{Var}(X)=\rho\text{Std}(X)\text{Std}(Y)-c\text{Var}(X)=\rho\sigma_1\sigma_2-c\sigma_1^2 = 0\\[8pt]
	\Rightarrow c &= \rho\,\frac{\sigma_2}{\sigma_1}
\end{align*}
	Since $X$ and $Y$ are independent r.v.s. with Normal, their linear combinations are still Normal, thus we know $(Y-cX, X)$ is Bivariate Normal.
	From Theorem 7.5.7. we know that when $c = \rho\,\frac{\sigma_2}{\sigma_1}$, $\ Y-cX$ and $X$ are independent.\vspace{1cm}
\question{7.79}
\Solution{}
(Let $D, R, N$ be r.v.s. of \#registered Democrats, \#registered Repubicans, \#people showed up at the polls)
\begin{enumerate}[(a)]
	\item We know that for a registered voter, the probability of showing up to the polls and being a Democrat is $ps$.
	By using the conclusion of Chicken-Egg story, $X$ is Poisson and $$X\sim \text{Pois}(ps\lambda)$$\vspace{0.4cm}
	\item By using the conclusion of Chicken-Egg story, $X|V$ is Binomial and $$X|V\sim \text{Bin}(v, ps)$$\vspace{0.4cm}
	\item Since the data of Repubicans does not affect $X$ due to the independence, we know that this is a Binomial with the probability of success is $s$,
	that is $$X|D, R  \sim \text{Bin}(d, s)$$\vspace{0.4cm}
	\item After knowing the true number of Democrats in those $v$ registered voters, that is $d$, we can update the proportion of Democrats to be $\frac{d}{v}$ instead of $p$.
	So we have the possibility of $\frac{d}{v}$ Democrats and selected $n$ people (i.e. who showed up) without replacement. So this is a Hypergeometric distribution that is $$X|D, R, N\sim\text{HGeom}(d, r, n)$$
\end{enumerate}



\end{document}