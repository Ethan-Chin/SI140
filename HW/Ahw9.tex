\documentclass[10.5pt]{article}
\usepackage{amsmath,amssymb,amsthm}
\usepackage{amsfonts,bm}
\usepackage{listings}
\usepackage{graphicx}
\usepackage[shortlabels]{enumitem}
\usepackage{tikz}
\usepackage{extramarks}
%\usepackage{enumerate}
\usepackage[margin=1in]{geometry}
\usepackage{fancyhdr}
\usepackage{epsfig}
\usepackage{amsmath}
\usepackage{float}
\usepackage{amssymb}
\usepackage{caption}
\usepackage{subfigure}
\usepackage{graphics}
\usepackage{titlesec}
\usepackage{mathrsfs}
\usepackage{amsfonts}
\usepackage{indentfirst}
\usepackage{fontspec}
\usepackage{dashrule}
\usepackage{ctex}
\usepackage{algpseudocode}
%\usepackage{algorithm}

\renewcommand{\baselinestretch}{1.2}

\usepackage{tikz-qtree}
\usetikzlibrary{graphs}
\tikzset{every tree node/.style={minimum width=2em,draw,circle},
	blank/.style={draw=none},
	edge from parent/.style=
	{draw,edge from parent path={(\tikzparentnode) -- (\tikzchildnode)}},
	level distance=1.2cm} 
\setlength{\parindent}{0pt}
\setlength{\headheight}{13.6pt}
\newcommand\question[1]{\vspace{.2in}\hrule\vspace{0.04in}\textbf{Problem\ #1}\vspace{.4em}\hrule\vspace{.10in}}
\newcommand\Solution{\vspace{.3in}\textbf{Solution:}\vspace{.5em}\hrule\vspace{.08in}\par}
\newcommand\Answer{\vspace{.2in}\textbf{Answer:}\vspace{.5em}\hrule\vspace{.08in}\par}
\newcommand\Proof{\vspace{.3in}\textbf{Proof:}\vspace{.5em}\hrule\vspace{.08in}\par}
\newcommand\minsolution{\vspace{.3in}\textbf{Solution:}\vspace{.4em}\par}
\newcommand\minanswer{\vspace{.2in}\textbf{Answer:}\vspace{.4em}\par}
\newcommand\minproof{\vspace{.3in}\textbf{Proof:}\vspace{.4em}\par}
\renewcommand\part[1]{\vspace{.10in}\textbf{(#1)}}
\newcommand\algorithm{\vspace{.10in}\textbf{Algorithm: }}
\newcommand\correctness{\vspace{.10in}\textbf{Correctness: }}
\newcommand\runtime{\vspace{.10in}\textbf{Running time: }}
\pagestyle{fancyplain}

\setCJKfamilyfont{Song}[AutoFakeBold]{SimSun}
\newcommand*{\Song}{\CJKfamily{Song}}




\newcommand{\horrule}[1]{\rule{\linewidth}{#1}}

\title{
	\normalfont \normalsize
	\begin{figure}[!h]
	\centering
	\includegraphics[width=4.8in, keepaspectratio]{logo_red.pdf}\\[1cm]
		%\caption{}
	\end{figure}
	%\huge{\textsc{ShanghaiTech University}} \\ [8pt]
	\horrule{0.5pt} \\[0.4cm]
	\Huge SI140 Probability \& Mathematical Statistics\\[0.4cm]
	\LARGE Homework 9\\
	\horrule{2pt} \\[1.5cm]
}
 
\author{\Song{\huge\textbf{陈昱聪}}\\[0.2cm]Chen Yucong\ ><E<>N\\[4.5cm]\textbf{Student ID: 2019533079}\\[0.2cm] 
\textbf{Email:}\ {\ttfamily chenyc@shanghaitech.edu.cn}\\[0.8cm] \LARGE\textsc{School of Information Science and Technology}\\[0.63cm]
\texttt{$\circledcirc$ Group\#2\ (TA:曾理)}}
\date{}


\pagestyle{fancy}
\lhead{SI140 Probability \& Mathematical Statistics}
\chead{\textbf{Homework 9\ }}
\rhead{陈昱聪\,2019533079\ \,Due:\,11:59\,am, $30^{\text{th}}$ Nov.}
\cfoot{\thepage}
\renewcommand{\headrulewidth}{0.4pt}


\fancypagestyle{firstpage}
{
	\renewcommand{\headrulewidth}{0pt}
	\fancyhf{}
	\fancyfoot[C]{\thepage}
}


\newcounter{ProblemCounter}
\newcounter{oldvalue}
\newcommand{\problem}[2][-1]{
	\setcounter{oldvalue}{\value{secnumdepth}}
	\setcounter{secnumdepth}{0}
	\ifnum#1>-1
	\setcounter{ProblemCounter}{0}
	\else
	\stepcounter{ProblemCounter}
	\fi
	\section{Problem \arabic{ProblemCounter}: #2}
	\setcounter{secnumdepth}{\value{oldvalue}}
}
\newcommand{\subproblem}[1]{
	\setcounter{oldvalue}{\value{section}}
	\setcounter{section}{\value{ProblemCounter}}
	\subsection{#1}
	\setcounter{section}{\value{oldvalue}}
}

\setmonofont{Consolas}
\definecolor{blve}{rgb}{0.3372549 , 0.61176471, 0.83921569}
\definecolor{gr33n}{rgb}{0.29019608, 0.7372549 , 0.64705882}
\makeatletter
\lst@InstallKeywords k{class}{classstyle}\slshape{classstyle}{}ld
\makeatother
\lstset{language=C++,
	basicstyle=\ttfamily,
	keywordstyle=\color{blve}\ttfamily,
	stringstyle=\color{red}\ttfamily,
	commentstyle=\color{green}\ttfamily,
	morecomment=[l][\color{magenta}]{\#},
	classstyle = \bfseries\color{gr33n}, 
	tabsize=4
}


\begin{document}
	
\maketitle
\thispagestyle{firstpage}
\thispagestyle{empty}
\setcounter{page}{0}



\question{7.38}
\Solution{}
Since $X$, $Y$, max$(X, Y)$ and min$(X, Y)$ are all r.v.s, we know that $X$ can be min$(X,Y)$ or max$(X,Y)$. When $X$ is 
min$(X,Y)$, $Y$ can only take max$(X, Y)$ ans vice versa. So the sum of them are the same that is max$(X, Y)+$min$(X, Y)=X+Y$.

However, COV(max($X, Y$), min($X, Y$))$\neq$Cov($X, Y$) since COV$(X)$ cannot be the COV(max($X, Y$)) nor COV(min($X, Y$)) and so can't $Y$.
And apparently, max$(X, Y)\geqslant$min$(X, Y)$ while the $X$ and $Y$ are not certain in who is greater. So the change-range of $X-Y$ will be sure greater than max$(X,Y)-$min$(X,Y)$.

\vspace{0.25cm}

\question{7.48}
\Solution{}
From the chicken-egg story, $X\sim$Pois$(\lambda p)$ and $X\sim$Pois$(\lambda q)$, $X$ and $Y$ are independent. So we have
$$\text{COV}(N, X)=\text{COV}(X+Y, X)=\text{COV}(X, X)+\text{COV}(Y, X)=\text{Var}(X)+0=\lambda p$$

So we have:
$$\text{Corr}(N, X)=\frac{\text{COV}(N, X)}{\sqrt{\text{Var}(N)\text{Var}(X)}}=\sqrt{p}$$

\vspace{0.25cm}


\question{7.53}
\Solution{}
Since we know that $X-Y\sim N(0, 2)$. Let $X-Y = \sqrt{2}Z,\Rightarrow  Z\sim N(0, 1)$ and $E(|X-Y|)=\sqrt{2}E|Z|$. 
$$E|Z| = \int_{-\infty}^{\infty}|z|\frac{1}{\sqrt{2\pi}}e^{-z^2/2}\,\mathrm{d}z=2\int_{0}^{\infty}z\frac{1}{\sqrt{2\pi}}e^{-z^2/2}\,\mathrm{d}z=\sqrt{\frac{2}{\pi}}$$
So we have $E(M)-E(L) = E(|X-Y|)=\frac{2}{\sqrt{\pi}}$. And we have
$E(M)+E(L)=E(X)+E(Y)=0$.

So that $E(M) = -E(L)=\frac{1}{\sqrt{\pi}}$. We have:
$$\text{Cov}(M, L)=E(ML) - E(M)E(L)=E(XY)-E(M)E(L)=0-E(M)E(L)=\frac{1}{\pi}$$
By symmetry
$$\text{Var}(M) = \text{Var}(L)=E(M^2) - E^2(M)=E(M^2) - \frac{1}{\pi}$$

Since $E((X-Y)^2) = 0$, we have $$\text{Var}(X-Y) = E^2(X-Y) = E^2(M-L) = E^2(M)+E^2(L) - 2E(M)E(L) = E^2(M)+E^2(L) = 2$$
So that $\text{Var}(M) = 1-\frac{1}{\pi}$, we have $$\text{Corr}(M, L)=\frac{\text{Cov}(M, L)}{\sqrt{\text{Var}(M)\text{Var}(L)}}=\frac{1}{\pi - 1}$$


\pagebreak
\question{7.55}
\Solution{}
\begin{enumerate}[(a)]
	\item $$\text{Cov}(X, Y)=\text{Cov}(V, V)+\text{Cov}(V,Z)+\text{Cov}(V, W)+\text{Cov}(Z, W)=\text{Var}(V, V) = \lambda$$\vspace{1cm}
	\item Since $\text{Cov}(X, Y)\neq0$, $X$ and $Y$ aree not independent.\begin{align*}
		P(X=x,Y=y|V=v)
		&=P(W=x-v, Z=y-z|V=v)\\[6pt]
		&=P(W=x-v, Z=y-z)\\[6pt]
		&=P(W = x-v)P(W = y-v)\\[6pt]
		&=P(X=x|V=v)P(Y=y|V=v)\\[6pt]
	\end{align*} So they are conditionally independent given $V$.\vspace{1cm}
	\item Let $L = \text{min}(X, Y)$, we get that $V\leqslant L$ \begin{align*}
		P(X=x, Y=y)&=\sum_{v=0}^{\infty}P(X=x, Y=y|V=v)P(V=v)\\[6pt]
		&=\sum_{v=0}^{L}P(X=x|V=v)P(Y=y|V=v)P(V=v)\\[6pt]
		&=\sum_{v=0}^{L}e^{-3\lambda}\frac{\lambda^{x-v}}{(x-v)!}\cdot \frac{\lambda^{y-v}}{(y-v)!}\cdot\frac{\lambda^v}{v!}\\[6pt]
		&=e^{-3\lambda}\sum_{v=0}^{L}\frac{\lambda^{x+y-v}}{(x-v)!(y-v)!v!}\\[6pt]
	\end{align*}For $X\geqslant0$,$Y\geqslant0$, $0$ otherwise. 
\end{enumerate}
\pagebreak

\question{7.57}
\Solution{}
\begin{enumerate}[(a)]
	\item Let $$X=\sum_{i = 1}^n x_iI_i,\quad Y=\sum_{j = 1}^ny_jJ_j$$
	\begin{align*}
		\text{Cov}(X, Y)
		&=\text{Cov}\left(\sum_{i = 1}^n x_iI_i, \sum_{j = 1}^ny_jJ_j\right)\\[6pt]
		&= \sum_{i, j = 1}(x_iy_j)\text{Cov}(I_i, J_j)\\[6pt]
		&= \sum_{i, j = 1}(x_iy_j)(E(I_iJ_j)-E(I_i)E(J_j))\\[6pt]
		&= \sum_{i, j = 1}(x_iy_j)(P(I_i = 1, J_j = 1)-P(I_i = 1)P(J_j = 1))\\[6pt]
		&= \frac{n-1}{n^2}\sum_{i=j}x_iy_j-\frac{1}{n^2}\sum_{i\neq j}x_iy_j\\[6pt]
	\end{align*}
	\begin{align*}
		r
		&= \frac{1}{n}\sum_{i=1}^n(x_i-\bar{x})(y_i-\bar{y})\\[6pt]
		&= \frac{1}{n}\sum_{i = 1}^n(x_iy_i-x_i\bar{y}-y_i\bar{x}+\bar{x}\bar{y})\\[6pt]
		&= \frac{1}{n}\sum_{i = 1}^nx_iy_i - \bar{x}\bar{y}\\[6pt]
		&= \frac{1}{n}\sum_{i = 1}^nx_iy_i - \frac{1}{n^2}\sum_{i = 1}^nx_i\cdot\sum_{i = 1}^ny_i\\[6pt]
		&= \frac{1}{n}\sum_{i = 1}^nx_iy_i - \frac{1}{n^2}\sum_{i = j}x_iy_j-\frac{1}{n^2}\sum_{i \neq j}x_iy_j\\[6pt]
		&=\frac{n-1}{n^2}\sum_{i=j}x_iy_j-\frac{1}{n^2}\sum_{i\neq j}x_iy_j\\[6pt]
	\end{align*}
	That is $\text{Cov}(X, Y) = r$.
	
	\item Since we get $n^2$ pairs of $(X, Y)$ and $(\widetilde{X}, \widetilde{Y})$. By symmetry, \begin{align*}
		total\ signed\ area &= \sum_{i, j}(y_j-y_i)(x_j-x_i)\\[6pt]
		&=n^2E((X-\widetilde{X})(Y-\widetilde{Y}))
	\end{align*}
	Also, \begin{align*}
		total\ signed\ area &= \sum_{i, j}(y_j-y_i)(x_j-x_i)\\[6pt]
		&= \sum_{i}\sum_j(x_jy_j - x_iy_j-x_jy_i+x_iy_j)\\[6pt]
		&=n\sum_ix_iy_i-n^2\bar{x}\bar{y}-n^2\bar{x}\bar{y}+n\sum_ix_iy_i\\[6pt]
		&= 2n\sum_ix_iy_i - 2n^2\bar{x}\bar{y} = 2n^2r
	\end{align*}\vspace{1.5cm}
	\item \begin{enumerate}[(i)]
		\item The order of Height and weight doesn't matter since the area of the rectangle won't change if we exchange the first coordinate and the second coordinate. And the covariance is just a constant times of the area.
		\item Scaling of coordinates will lead to the area scale being the product of the scaling factor of the two coordinates
		\item Shifting a rectangle doesn't change its area.
		\item The area when expanding one coordinate is equal to the sum of two areas of these little rectangles.
	\end{enumerate}
\end{enumerate}
\pagebreak
\question{7.58}
\Solution{}
\begin{enumerate}[(a)]
	\item $$E(\widehat{\theta}) = E(\omega_1\widehat{\theta}_1+\omega_2\widehat{\theta}_2)=E(\omega_1\widehat{\theta}_1)+E(\omega_2\widehat{\theta}_2)=(\omega_1+\omega_2)E(\widehat{\theta})=\theta$$\vspace{0.4cm}
	\item Since they are unbiased, \begin{align*}
		\frac{\mathrm{d}}{\mathrm{d} \omega_1} MSE(\widehat{\theta}) = \frac{\mathrm{d}}{\mathrm{d} \omega_1}\text{Var}(\widehat{\theta})
		&=\frac{\mathrm{d}}{\mathrm{d} \omega_1}(\omega_1^2\text{Var}(\widehat{\theta}_1)+\omega_2^2\text{Var}(\widehat{\theta}_2))\\[6pt]
		&=\frac{\mathrm{d}}{\mathrm{d} \omega_1}(\omega_1^2\text{Var}(\widehat{\theta}_1)+(1-\omega_1)^2\text{Var}(\widehat{\theta}_2))\\[6pt]
		&=\omega_1\text{Var}(\widehat{\theta}_1)-\omega_2\text{Var}(\widehat{\theta}_2)=0\\[6pt]
		&\Rightarrow \omega_1\text{Var}(\widehat{\theta}_1)=\omega_2\text{Var}(\widehat{\theta}_2)\\[6pt]
		&\Rightarrow \omega_1=\frac{\text{Var}(\widehat{\theta}_2)}{\text{Var}(\widehat{\theta}_1)+\text{Var}(\widehat{\theta}_2)}, \qquad\omega_2=\frac{\text{Var}(\widehat{\theta}_1)}{\text{Var}(\widehat{\theta}_1)+\text{Var}(\widehat{\theta}_2)}
	\end{align*}\vspace{0.4cm}
	\item Let the variance of them be $\sigma^2$.$$\text{Var}(\widehat{\theta}_1)=\text{Var}\left(\frac{1}{n}\sum_{i = 1}^nX_i\right) = \frac{1}{n^2}\sum_{i = 1}^n\text{Var}(X_i)=\frac{\sigma^2}{n}$$
	And thus $$\text{Var}(\widehat{\theta}_2)=\frac{\sigma^2}{m}$$
	So we get$$\omega_1=\frac{\text{Var}(\widehat{\theta}_2)}{\text{Var}(\widehat{\theta}_1)+\text{Var}(\widehat{\theta}_2)}=\frac{n}{m+n}$$
	$$\omega_2=\frac{\text{Var}(\widehat{\theta}_1)}{\text{Var}(\widehat{\theta}_1)+\text{Var}(\widehat{\theta}_2)}=\frac{m}{m+n}$$
	Finally we have\begin{align*}
		\widehat{\theta} = \widehat{\theta} = \omega_1\widehat{\theta}_1+\omega_2\widehat{\theta}_2
		&=\frac{n}{m+n}\widehat{\theta}_1+\frac{m}{m+n}\widehat{\theta}_2\\[6pt]
		&=\frac{n}{m+n}\left(\frac{1}{n}\sum_{i = 1}^nX_i\right)+\frac{m}{m+n}\left(\frac{1}{m}\sum_{i = 1}^mY_i\right)\\[6pt]
		&=\frac{1}{m+n}\left(\sum_{i = 1}^nX_i+\sum_{i = 1}^mY_i\right)
	\end{align*}
	Thus, $\widehat{\theta}$ is the mean of the whole sample.
\end{enumerate}

\end{document}