\documentclass[10.5pt]{article}
\usepackage{amsmath,amssymb,amsthm}
\usepackage{amsfonts,bm}
\usepackage{listings}
\usepackage{graphicx}
\usepackage[shortlabels]{enumitem}
\usepackage{tikz}
\usepackage{extramarks}
%\usepackage{enumerate}
\usepackage[margin=1in]{geometry}
\usepackage{fancyhdr}
\usepackage{epsfig}
\usepackage{amsmath}
\usepackage{float}
\usepackage{amssymb}
\usepackage{caption}
\usepackage{subfigure}
\usepackage{graphics}
\usepackage{titlesec}
\usepackage{mathrsfs}
\usepackage{amsfonts}
\usepackage{indentfirst}
\usepackage{fontspec}
\usepackage{dashrule}
\usepackage{ctex}
\usepackage{algpseudocode}
%\usepackage{algorithm}

\renewcommand{\baselinestretch}{1.2}

\usepackage{tikz-qtree}
\usetikzlibrary{graphs}
\tikzset{every tree node/.style={minimum width=2em,draw,circle},
	blank/.style={draw=none},
	edge from parent/.style=
	{draw,edge from parent path={(\tikzparentnode) -- (\tikzchildnode)}},
	level distance=1.2cm} 
\setlength{\parindent}{0pt}
\setlength{\headheight}{13.6pt}
\newcommand\question[1]{\vspace{.2in}\hrule\vspace{0.04in}\textbf{Problem\ #1}\vspace{.4em}\hrule\vspace{.10in}}
\newcommand\Solution{\vspace{.3in}\textbf{Solution:}\vspace{.5em}\hrule\vspace{.08in}\par}
\newcommand\Answer{\vspace{.2in}\textbf{Answer:}\vspace{.5em}\hrule\vspace{.08in}\par}
\newcommand\Proof{\vspace{.3in}\textbf{Proof:}\vspace{.5em}\hrule\vspace{.08in}\par}
\newcommand\minsolution{\vspace{.3in}\textbf{Solution:}\vspace{.4em}\par}
\newcommand\minanswer{\vspace{.2in}\textbf{Answer:}\vspace{.4em}\par}
\newcommand\minproof{\vspace{.3in}\textbf{Proof:}\vspace{.4em}\par}
\renewcommand\part[1]{\vspace{.10in}\textbf{(#1)}}
\newcommand\algorithm{\vspace{.10in}\textbf{Algorithm: }}
\newcommand\correctness{\vspace{.10in}\textbf{Correctness: }}
\newcommand\runtime{\vspace{.10in}\textbf{Running time: }}
\pagestyle{fancyplain}

\setCJKfamilyfont{Song}[AutoFakeBold]{SimSun}
\newcommand*{\Song}{\CJKfamily{Song}}




\newcommand{\horrule}[1]{\rule{\linewidth}{#1}}

\title{
	\normalfont \normalsize
	\begin{figure}[!h]
	\centering
	\includegraphics[width=4.8in, keepaspectratio]{logo_red.pdf}\\[1cm]
		%\caption{}
	\end{figure}
	%\huge{\textsc{ShanghaiTech University}} \\ [8pt]
	\horrule{0.5pt} \\[0.4cm]
	\Huge SI140 Probability \& Mathematical Statistics\\[0.4cm]
	\LARGE Homework 4\\
	\horrule{2pt} \\[1.5cm]
}

\author{\Song{\huge\textbf{陈昱聪}}\\[0.2cm]Chen Yucong\ ><E<>N\\[4.5cm]\textbf{Student ID: 2019533079}\\[0.2cm] 
\textbf{Email:}\ {\ttfamily chenyc@shanghaitech.edu.cn}\\[0.8cm] \LARGE\textsc{School of Information Science and Technology}\\[0.63cm]
\texttt{$\circledcirc$ Group\#2\ (TA:曾理)}}
\date{}


\pagestyle{fancy}
\lhead{SI140 Probability \& Mathematical Statistics}
\chead{\textbf{Homework 4}}
\rhead{陈昱聪\ 2019533079\quad Due: 11:59\,am $10^{\text{th}}$ Oct.}
\cfoot{\thepage}
\renewcommand{\headrulewidth}{0.4pt}


\fancypagestyle{firstpage}
{
	\renewcommand{\headrulewidth}{0pt}
	\fancyhf{}
	\fancyfoot[C]{\thepage}
}


\newcounter{ProblemCounter}
\newcounter{oldvalue}
\newcommand{\problem}[2][-1]{
	\setcounter{oldvalue}{\value{secnumdepth}}
	\setcounter{secnumdepth}{0}
	\ifnum#1>-1
	\setcounter{ProblemCounter}{0}
	\else
	\stepcounter{ProblemCounter}
	\fi
	\section{Problem \arabic{ProblemCounter}: #2}
	\setcounter{secnumdepth}{\value{oldvalue}}
}
\newcommand{\subproblem}[1]{
	\setcounter{oldvalue}{\value{section}}
	\setcounter{section}{\value{ProblemCounter}}
	\subsection{#1}
	\setcounter{section}{\value{oldvalue}}
}

\setmonofont{Consolas}
\definecolor{blve}{rgb}{0.3372549 , 0.61176471, 0.83921569}
\definecolor{gr33n}{rgb}{0.29019608, 0.7372549 , 0.64705882}
\makeatletter
\lst@InstallKeywords k{class}{classstyle}\slshape{classstyle}{}ld
\makeatother
\lstset{language=C++,
	basicstyle=\ttfamily,
	keywordstyle=\color{blve}\ttfamily,
	stringstyle=\color{red}\ttfamily,
	commentstyle=\color{green}\ttfamily,
	morecomment=[l][\color{magenta}]{\#},
	classstyle = \bfseries\color{gr33n}, 
	tabsize=4
}


\begin{document}
	
\maketitle
\thispagestyle{firstpage}
\thispagestyle{empty}
\setcounter{page}{0}

\pagebreak

\question{2.38}
	\Solution{}
    \begin{enumerate}[(a)]
        \item I pick the door $1$.\quad Monty open the door $2\sim 4$.
	$$M_j^i = ``\text{Monty open the door}\ i\sim j"\qquad C_k = ``\text{Car is behind the}\ k^{\text{th}}\ \text{door}"$$
	$$P(C_k) = \frac{1}{7},\quad (k = 1, 2, \dots, 7)$$
	\begin{align*}
\text{LOTP:}\qquad P(M^2_4) &= P(M^2_4|C_1)P(C_1)+\sum_{i = 2}^{4}P(M^2_4|C_i)P(C_i)+\sum_{i = 5}^{7}P(M^2_4|C_i)P(C_i)\\[8pt]
&=\frac{1}{\binom{6}{3}}\cdot\frac{1}{7}+0+\frac{1}{\binom{5}{3}}\cdot\frac{1}{7}\cdot3\\[8pt]
&=\frac{1}{20}
\end{align*}
\begin{align*}
                P(C_1|M^2_4) &= \frac{P(M^2_4|C_1)P(C_1)}{P(M^2_4)} = \frac{1}{7}\\[8pt]
                P(C_5|M^2_4) &=P(C_6|M^2_4)=P(C_7|M^2_4) = \frac{P(M^2_4|C_5)P(C_5)}{P(M^2_4)} = \frac{2}{7}>P(C_1|M^2_4)
        \end{align*}
        I should switch, the probability of switch to one of the remaining $3$ doors is $\frac{2}{7}$.

\vspace{1cm}
        \item I pick the door $1$.\quad Monty open the door $2\sim m+1$.
	$$M_{m+1}^2 = ``\text{Monty open the door}\ 2\sim m+1"\qquad C_k = ``\text{Car is behind the}\ k^{\text{th}}\ \text{door}"$$
	$$P(C_k) = \frac{1}{n},\quad (k = 1, 2, \dots, n)$$
	\begin{align*}
\text{LOTP:}\qquad P(M^2_{m+1}) &= P(M^2_{m+1}|C_1)P(C_1)+\sum_{i = 2}^{m+1}P(M^2_{m+1}|C_i)P(C_i)+\sum_{i = m+2}^{n}P(M^2_{m+1}|C_i)P(C_i)\\[8pt]
&=\frac{1}{\binom{n-1}{m}}\cdot\frac{1}{n}+0+\frac{1}{\binom{n-2}{m}}\cdot\frac{1}{n}\cdot(n-m-1)\\[8pt]
&=\frac{m!}{(n-1)(n-2)\dots(n-m)}
\end{align*}
\begin{align*}
                P(C_1|M^2_{m+1}) &= \frac{P(M^2_{m+1}|C_1)P(C_1)}{P(M^2_{m+1})} = \frac{1}{n}\\[8pt]
				P(C_i|M^2_{m+1}) &=\frac{P(M^2_{m+1}|C_i)P(C_i)}{P(M^2_{m+1})} = \frac{1}{n}\cdot\frac{n-1}{n-m-1}>P(C_1|M^2_{m+1})\quad\text{For $i = m+2, m+3\dots, n$}
        \end{align*}
        I should switch, the probability of switch to one of the remaining doors is $\frac{1}{n}\cdot\frac{n-1}{n-m-1}$.
    \end{enumerate}
		
       
\vspace{2cm}

\pagebreak

\question{2.39}
	\Solution{}
	\begin{enumerate}[(a)]
		\item $$M_i = ``\text{Monty open the door}\ i"\quad C_k = ``\text{Car is behind the}\ k^{\text{th}}\ \text{door}"$$ $$P(C_k) = \frac{1}{3},\quad k = 1, 2\ \text{or}\ 3$$
		$$P(M_2) = P(M_2|C_1)P(C_1)+P(M_2|C_2)P(C_2)+P(M_2|C_3)P(C_3) =\frac{1}{3}\cdot p + 0 + \frac{1}{3}\cdot1= \frac{p+1}{3}$$
		$$P(M_3) = 1-P(M_2)= \frac{2-p}{3}$$
		\begin{align*}
			P(C_1|\text{After Monty chose}) 
			&= P(C_1|M_2)P(M_2)+P(C_1|M_3)P(M_3)\\[8pt]
			&= P(M_2|C_1)P(C_1)+P(M_3|C_1)P(C_1) = \frac{1}{3}\\[8pt]
			P(\text{Win on switching}) &= 1 - P(C_1|\text{After Monty chose}) = \frac{2}{3}
		\end{align*}\vspace{1cm}
		\item \begin{align*}
			P(\text{Win on switching}|M_2) = P(C_3|M_2) = \frac{P(M_2|C_3)P(C_3)}{P(M_2)} = \frac{1}{p+1}
		\end{align*}\vspace{1cm}
		\item \begin{align*}
			P(\text{Win on switching}|M_3) = P(C_2|M_3) = \frac{P(M_3|C_2)P(C_2)}{P(M_3)} = \frac{1}{2-p}
		\end{align*}
	\end{enumerate}

\pagebreak

\question{2.42}
	\Solution{}
	\begin{enumerate}[(a)]
		\item We can say that the running total would never be negative. So we define $p_k = 0$ for $k<0$. And it is clear that the running total reached to $0$ before the very first rolling
		so that $p_0 = 1$. The value of $p_n$ is that the last time reached to $n - 1$ and this time got an $1$; or the last time reached to $n - 2$ and this time got a $2$, and so on.
		So we get $$p_n = \frac{1}{6}(p_{n - 1}+p_{n - 2}+p_{n - 3}+p_{n - 4}+p_{n - 5}+p_{n - 6})\qquad n\in\mathbb{N}_+$$
		\vspace{2cm}


		\item \begin{align*}
			&p_0 = 1\qquad p_1 = \frac{1}{6}\qquad p_2 = \frac{1}{6}(1+\frac{1}{6})\\[8pt]
			&p_3 = \frac{1}{6}(1+\frac{1}{6}+\frac{1}{6}(1+\frac{1}{6}))=\frac{1}{6}(1+\frac{1}{6})^2\\[8pt]
			&p_4 = \frac{1}{6}(1+\frac{1}{6})^3\qquad p_5 = \frac{1}{6}(1+\frac{1}{6})^4\qquad p_6 = \frac{1}{6}(1+\frac{1}{6})^5\\[8pt]
			&p_7 = \frac{1}{6}(p_1+p_2+p_3+p_4+p_5+p_6) = \frac{1}{6}\left((1+\frac{1}{6})^6-1\right) \approx 0.2536
		\end{align*}

		\vspace{2cm}


		\item On average, the number thrown is $\frac{1+2+3+4+5+6}{6} = \frac{7}{2}$. That means the growth speed of the running total is $3.5/\text{times}$. That means every time we roll two times, the running total goes up over $7$ numbers but we only get $2$ results during the 2 rolls.
		
		So the probability when $n$ is big enough is $\frac{2}{7}$
	\end{enumerate}
		
\pagebreak

\question{2.52}
\Proof{}
\begin{enumerate}[(a)]
	\item \textbf{Impossible.} 
	
	If $A$ and $B$ are independent, then $P(A|B) = P(A) = P(A|B^c)$.
	
	$\Box$
	\vspace{2cm}
	\item \textbf{Impossible.} \begin{align*}
		P(A) = P(A|C) 
		&= P(A|B, C)P(B)+P(A|B^c, C)P(B^c)\\[8pt]
		&< P(A|B^c, C)P(B)+P(A|B^c, C)P(B^c)\\[8pt]
		& = P(A|B^c, C)
	\end{align*}
	Similarly, we have $$P(A)<P(A|B^c, C^c)$$
	So that $$P(A)<P(A|B^c, C)P(C)+P(A|B^c, C^c)P(C^c) = P(A|B^c)$$
	Similarly, we have $$P(A)>P(A|B)$$
	So that $$P(A|B)<P(A|B^c)$$ $\Box$
\vspace{2cm}
	\item \textbf{Impossible.}\begin{align*}
		P(A|B) 
		&= P(A|B, C)P(C|B)+P(A|B, C^c)P(C^c|B)\\[8pt]
		&= P(A|B, C)P(C|B^c)+P(A|B, C^c)P(C^c|B^c)\\[8pt]
		&< P(A|B^c, C)P(C|B^c)+P(A|B^c, C^c)P(C^c|B^c)\\[8pt]
		&= P(A|B^c)
	\end{align*}

	$\Box$
\end{enumerate}

\pagebreak

\question{2.63}
\begin{enumerate}[(a)]
	\item \minproof{}
	$$A_n = ``\text{Treatment $A$ is assigned on the $n^{th}$ trial}"\quad S_n = ``\text{The success on the $n^{th}$ trial}"$$
	\begin{align*}
		p_n &= P(S_n) = P(S_n|A_n)P(A_n)+P(S_n|A^c_n)P(A_n^c) = a\cdot a_n+b\cdot(1-a_n) = (a-b)a_n+b\\[8pt]
		a_{n+1} &= P(A_{n+1}) = P(A_n\cap S_n)+P(A_n^c\cap S_n^c)=P(S_n|A_n)P(A_n)+P(S_n^c|A_n^c)P(A_n^c)\\[8pt]
		&= (a+b-1)a_n+a-b
	\end{align*}
	$\Box$
	\item \minproof{}
	\begin{align*}
		p_n=(a-b)a_n+b
		&\Rightarrow a_n = \frac{p_n-b}{a-b}\\[8pt]
		&\Rightarrow a_{n+1} = (a+b-1)\frac{p_n-b}{a-b}+1-b\\[8pt]
		&\Rightarrow p_{n+1} = (a-b)a_{n+1}+b = (a-b)\left[(a+b-1)\frac{p_n-b}{a-b}+1-b\right]+b\\[8pt]
		&\Rightarrow p_{n+1} = (a+b-1)p_n+a+b-2ab
	\end{align*}
	$\Box$
	\item \minsolution{}Assume$$p = \lim_{n \to \infty} p_n = \lim_{n \to \infty} p_{n+1}$$
	Then we get
	\begin{align*}
		p_{n+1} &= (a+b-1)p_n+a+b-2ab\\[8pt]
		\Rightarrow p &= \frac{a+b-2ab}{2-a-b}
	\end{align*}
\end{enumerate}


\pagebreak

\question{2.64}
\Solution{}
\begin{enumerate}[(a)]
	\item \begin{align*}
		P(\text{child is $AA$}) 
		&= P(\text{child is $AA$}|\text{parents are both $AA$})P(\text{parents are both $AA$})\\[8pt]
		&+P(\text{child is $AA$}|\text{parents are both $Aa$})P(\text{parents are both $Aa$})\\[8pt]
		&+P(\text{child is $AA$}|\text{parents are $AA$ and $Aa$})P(\text{parents are $AA$ and $Aa$})\\[8pt]
		&=p^4+p^2(1-p)^2+2p^3(1-p) = p^2
	\end{align*}
	Similarly, $$P(\text{child is $Aa$}) = 2p(1-p)\qquad P(\text{child is $aa$}) = (1-p)^2$$
	So that the frequencies of genotypes don't change through generations after we reaches the Hardy-Weinberg balance. So the law is stable.
	\vspace{1cm}
	\item $$M = ``\text{child is homozygous}"\qquad N = ``\text{parents are homozygous}"$$
	\begin{align*}
		P(M|N)
		&= P(\text{parents are both $AA$}|N)+P(\text{parents are both $aa$}|N)\\[8pt]
		&= \frac{P(N|\text{parents are both $AA$})P(\text{parents are both $AA$})}{P(N)}\\[8pt]
		&+\frac{P(N|\text{parents are both $aa$})P(\text{parents are both $aa$})}{P(N)}\\[8pt]
		&=\frac{P(\text{parents are both $AA$})+P(\text{parents are both $aa$})}{P(N)}\\[8pt]
		&=\frac{p^4+(1-p)^4}{(p^2+(1-p)^2)^2}\\[8pt]
	\end{align*}
	\begin{align*}
		P(\text{child is $Aa$}|\text{parents are $Aa$})= 2\cdot \frac{1}{2}\cdot\frac{1}{2} = \frac{1}{2}
	\end{align*}
	\pagebreak
	\item \begin{align*}
		&P(\text{child is heterozygous}|\text{no parent is $aa$})\\[8pt]
		=& P(\text{child is $Aa$}|\text{no parents is $aa$})\\[8pt]
		=& P(\text{child is $Aa$}|\text{parents are $AA$ and $Aa$})P(\text{parents are $AA$ and $Aa$})\\[8pt]
		+& P(\text{child is $Aa$}|\text{parents are both $Aa$})P(\text{parents are both $Aa$})\\[8pt]
		+& P(\text{child is $Aa$}|\text{parents are both $AA$})P(\text{parents are both $AA$})\\[8pt]
		=&\frac{1}{2}\cdot2\cdot p^2\cdot2p(1-p)+\frac{1}{2}\cdot\left[2p(1-p)\right]^2+0\\[8pt]
		=& 2p^2(1-p)\\[8pt]
	\end{align*}
	Similarly, 
	\begin{align*}
		P(\text{child is not $aa$}|\text{no parent is $aa$}) = p^2(3-2p)\\[8pt]
	\end{align*}
	\begin{align*}
		&P(\text{child is heterozygous}|\text{no parent is $aa$},\text{child is not $aa$})\\[8pt]
		=&\frac{P(\text{child is not $aa$}|\text{child is heterozygous}, \text{no parent is $aa$})P(\text{child is heterozygous}|\text{no parent is $aa$})}{P(\text{child is not $aa$}|\text{no parent is $aa$})}\\[8pt]
		=&\frac{P(\text{child is heterozygous}|\text{no parent is $aa$})}{P(\text{child is not $aa$}|\text{no parent is $aa$})}\\[8pt]
		=&\frac{2-2p}{3-2p}
	\end{align*}
\end{enumerate}
\end{document}